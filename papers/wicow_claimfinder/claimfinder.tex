\documentclass{acm_proc_article-sp}
\usepackage{times}
\usepackage{url}
\usepackage{graphics}
\usepackage{color}
\newcommand{\want}[1]{}
\newcommand{\idea}[1]{}
\newcommand{\note}[1]{}
\newcommand{\x}[1]{}
\newcommand{\todo}[1]{}
\newcommand{\maybe}[1]{}
\newcommand{\resolved}[1]{}

% Mark up a point that we want to flesh out into more text
%\newcommand{\x}[1]{{\color{blue} #1}\\}

% Mark up some work that we need to do in order for this paper to be telling the truth
%\newcommand{\todo}[1]{{\color{red} #1}\\}

%\newcommand{\maybe}[1]{}

\begin{document}

\toappear

\bibliographystyle{plain}

\title{What is disputed on the web?}

%%
%% Note on formatting authors at different institutions, as shown below:
%% Change width arg (currently 7cm) to parbox commands as needed to
%% accommodate widest lines, taking care not to overflow the 17.8cm line width.
%% Add or delete parboxes for additional authors at different institutions. 
%% If additional authors won't fit in one row, you can add a "\\"  at the
%% end of a parbox's closing "}" to have the next parbox start a new row.
%% Be sure NOT to put any blank lines between parbox commands!
%%

\numberofauthors{5}

\author{
\alignauthor Rob Ennals\\
       \affaddr{Intel Labs Berkeley}\\
       \affaddr{2150 Shattuck Ave}\\
       \affaddr{Berkeley, CA, USA}\\
       \email{robert.ennals@intel.com}
\alignauthor Dan Byler\\
       \affaddr{School of Information}\\
       \affaddr{University of California at Berkeley}\\
       \affaddr{Berkeley, CA, USA}\\
       \email{daniel.byler@berkeley.edu}
\and
\alignauthor John Mark Agosta\\
       \affaddr{Intel Labs Santa Clara}\\
       \affaddr{2200 Mission College Blvd}\\
       \affaddr{Santa Clara, CA, USA}\\
       \email{john.m.agosta@intel.com}
\alignauthor Barbara Rosario\\
       \affaddr{Intel Labs Santa Clara}\\
       \affaddr{2200 Mission College Blvd}\\
       \affaddr{Santa Clara, CA, USA}\\
       \email{john.m.agosta@intel.com}
}

%\sloppy


\maketitle

%RULE: Don't cite media reports unless I have to - some reviewers don't like it


\abstract

We present a method for the automatic acquisition of a corpus of disputed claims from the web. We consider a claim to be disputed if a page on the web suggests that this claim is wrong and that other people on the web are suggesting that it is true.

Our tool extracts disputed claims by searching the web for patterns such as ``falsely claimed that $X$'' and then using a statistical classifier to select text that appears to be making a disputed claim. 

We argue why such a corpus of disputed claims is useful for a wide range of applications related to information credibility on the web, and we look at what our current corpus reveals about what is being disputed on the web.

\resolved{too much emphasis on {\em how}... rewrite following paragrapgh}

%\subsection{Categories and Subject Descriptors}
%\todo{Categories, terms, and keywords need updating}
\category{H.3.1}{INFORMATION STORAGE AND RETRIEVAL}{Content Analysis and Indexing}
\category{I.2.7}{ARTIFICIAL INTELLIGENCE}{Natural Language Processing}

\terms{Design, Human Factors}

\keywords{Sensemaking, Annotation, Argumentation, Web, CSCW}

\tolerance=400 
  % makes some lines with lots of white space, but  
  % tends to prevent words from sticking out in the margin

\section{Introduction}

The web contains a vast number of pages written by a vast number of people. In many cases, these people disagree with each other, and the pages they write contain conflicting information. For users to extract reliable information from the web, it is useful for them to be able to determine when different sources disagree about a topic.

In this paper we describe a method for automatically acquiring a corpus of disputed claims from the web. We consider a claim to be {\it disputed} if a page exists on the web that suggests both that the claim is false, and that there are others that are making the claim. We are also interested in {\it who} disputes a particular claim, since a user is likely to be more interested in being told that a claim is disputed by a large number of sources that they believe are credible, rather than a small number of sources that they do not trust.

We identify disputed claims by searching for lexical patterns such as ``the misconception that $S$'' (Section~\ref{findingclaims}). For example, if a page contains the text ``the misconception that {\it the moon is made of cheese}'', this suggests that the author believes that others are saying that the moon is made of cheese, and that the author themselves disputes this claim. In Figure~\ref{templates} we list some of the patterns we used. In practice, many of the strings that match such a pattern are not well-formed disputed claims. We use a statistical classifier to filter out those strings that appear to make a well-formed claim.

We believe that this corpus will be useful for many purposes. We are currently using it as part of our Dispute Finder~\cite{Ennals2010} project to automatically highlight phrases on web pages that are disputed by other sources. We are also exploring other applications of this corpus, including automatically detecting disputed claims in human speech, visualization tools that let users know what is disputed about a topic that interests them, and statistical tools that allow people to look for patterns in internet debate  (Section~\ref{using}).

We have made our corpus publicly available for other researchers to download at \url{http://confront.intel-research.net/}. It currently contains claims extracted from pages written on the days between October 1st 2009 and January 19th 2010, and pages written on January 10th of every year from 2000 to 2010. This amounts to approxmately 1.1 million strings that we believe are making disputed claims. We plan to expand our corpus as we crawl more of the web and improve our algorithms.

By looking at the claims in this corpus, we see trends in dispute on the web that correspond to the debates that were particulary topical at particular times (Section~\ref{analysis}).

We believe that ours is the first attempt to automatically acquire a corpus of disputed claims from the Web.


\section{Background and Related Work}

The problem of determining information credibility on the web is becoming increasingly important. In the past a user would typically obtain information from a relatively small set of sources such as books, TV channels, and radio stations. The user would have some idea of the reputation and biases of each of these sources, and publishing barriers and quality control would ensure that these sources only published information that met their standards. 
The web is very different. A user has access to a vast number of different sources, but knows little about the biases or reputation of most of them. Moreover, the internet has little in the way of publishing barriers or quality control. If a user is to extract reliable information from the web, they need to either restrict themselves to a small set of trusted sources, or use some kind of mechanism to determine the credibility of the information that they access.

There are several ways that a user can determine whether to trust information that they find on a web page. The user can check the reputation of the person or organization that wrote the web page; they can observe whether the web page meets their aesthetic criteria for a reliable web page; or they can check whether the information is consistent with information available from other sources.

A user can check the reputation of a source using a service such as SourceWatch.org, which publishes manually curated information about the reputation and known biases of various sources. Trustpilot.com produce a Firefox extension that warns a user when they are looking at a web page hosted by a company that they believe is not trustworthy. Alternatively, a user can simply use a search engine such as Google to look for information about the source. While these tools can be very useful, trustworthy sources sometimes publish unreliable information, and untrusted sources sometimes contain useful information. For example, a reliable source may have been misled by an unreliable source they were using themselves; or a small unrated blog may publish information that useful, insightful, and accurate. 

Researchers have identified a variety of metrics that can be used to automatically estimate the quality of a document based on looking at its content. For Wikipedia, Blumenstock~\cite{Blumenstock2008} estimates the quality of an article by the word count and WikiTrust~\cite{Adler2008b} identifies potentially unreliable sections of an article by analyzing its edit history. Custard and Sumner~\cite{Custard2005} use a combination of metrics to measure web site quality, including number of links and whether it contains videos. Fogg et al~\cite{Fogg2003} have shown that users commonly evaluate the credibility of a web site based on factors such as the design look, the information structure of the site, and the tone of the writing. Some of the factors identified by Fogg et al could potentially be measured automatically and used to guide a user. These metrics do a good job at detecting pages that resemble pages that contain unreliable information, but they do not protect against authors who write untrustworthy information in the style of a trustworthy document.

In our research, we are following a third approach. We aim to inform users when information that they encounter is disputed by another source. We have built an extension to the Firefox web browser called Dispute Finder~\cite{Ennals2010} that informs users when a web page that they are reading makes a claim that it knows to be disputed. For example, if a user is reading a page that says ``Elvis is Alive'' then Dispute Finder will highlight that statement as being disputed and direct the user to other sources that put forward alternative points of view. In its currently released version, Dispute Finder builds a corpus of disputed claims by allowing users to enter disputed claims manually, and scraping a small set of web sites that manually curate such claims (currently Politifact.org and Snopes.com); however it is difficult to make this approach scale to the huge number of claims on the web that are disputed.

Our primary motivation for automatically acquiring a large corpus of disputed claims is to use this corpus to enable tools like Dispute Finder to automatically inform users when they encounter information that this corpus says is disputed. However, as we discuss in Section~\ref{using}, we believe such a corpus could be useful for many other purposes too.

The problem of detecting disputed claims is closely related to the well-studied problem of detecting contradictions. A claim $H$ is {\it contradicted} if a document somewhere on the web makes a claim that implies that $H$ cannot be true. For example, the claim ``The moon is made of rock'' {\it contradicts} the claim ``The moon is made of cheese''. A contradicted claim is not necessarily a disputed claim. In this example the person who said that the moon was made of rock might not be aware that others think it is made of cheese. A {\it contradiction} is logical, while a {\it dispute} is social. In order for a claim to be disputed, the authors of the documents must be aware that there is a contradiction between their beliefs.

There has been significant work on detecting contradictions. Condoravdi et al~\cite{Condoravdi2003} argue that contradiction detection is one of the key tasks in language understanding. De Marneffe et al~\cite{deMarneffe2008} present a taxonomy of the different ways that claims can contradict each other and describe a system that combines many techniques to detect different kinds of contradictions. AuContraire~\cite{Ritter} uses TextRunner~\cite{Etzioni2008,Banko2008} to infer subject-verb-object relationships, looks for cases where a verb maps the same subject to multiple objects, and uses semantic knowledge to determine when this implies that there is a contradiction. Harabagiu et al~\cite{Harabagiu2006} look for contradictions where one claim is a negated paraphrase of another. The RTE-3 Recognizing Textual Entailment challenge~\cite{Giampiccolo2007} included an optional contradiction detection task, allowing different groups building contradiction detection algorithms to compare their results. 

Detecting contradictions has proven to be a hard problem~\cite{Giampiccolo2007}. Much of this difficulty comes from the fact that one typically needs deep semantic and contextual knowledge to determine whether two statements that look like they might contradict each other actually do. For example ``George Bush is married to Barbara Bush'' does not contradict ``George Bush is married to Laura Bush'' because there is more than one George Bush. ``Alan Turing was born in England'' does not contradict ``Alan Turing was born in London'' because London is in England. ``It is raining is San Francisco'' does not contradict ``It is not raining in San Francisco'' if the statements were made at different times. 

We detect disputes rather than contradictions. Rather than looking for claims that contradict each other, we look for evidence that people believe that there is a contradiction and that this contradiction is important. One advantage of looking for disputes rather than contradictions is that this allows humans to do the hard work of identifying contradictions and deciding whether they are important. If a page says ``Falsely claimed that George Bush is married to Barbara Bush'' then that suggests that the contradiction with ``George Bush is married to Laura Bush'' is likely to be genuine. The corresponding disadvantage is that if one looks for disputes, one will only detect contradictions that humans have found and believe are important.

Perhaps more importantly, for our purposes disputes are more interesting than contradictions. We are interested in the social side of dispute. We want to know who disagrees with who, why they disagree, and what they think is important. 

Another closely related area is sentiment analysis/opinion mining~\cite{Hu2004,Pang2004}. Opinion mining tries to determine what an author's opinion is about certain objects or certain features of certain objects. A focal application has been the automatic summarisation of product reviews, to produce an overview of the product features that are viewed favorably or negatively. Dispute mining could be thought of as opinion mining applied to beliefs and ideas, rather than features of objects.

\resolved{All merged? : The methods we use are closely related to related work on information extraction which is the task of identifing and classifing specific semantic entities within documents (i.e. names of locations, people, organizations, temporal expressions etc.) We described such related work in the next Section.}


\section{Finding Disputed Claims}
\label{findingclaims}

\begin{figure}[tb]
\begin{center}
  \begin{tabular}{|llll|}
    \hline
    {\bf Frequency$^*$} & {\bf Pattern} & {\bf Pattern} & {\bf Classifier}\\ 
    {\bf on the web} &                   & {\bf Precision} & {\bf Accuracy}\\
    \hline
        \multicolumn{2}{|l}{\it Plausible to not believe it} & &\\
	    & & &\\

        570,000,000 & think that & 28\%& 69\%\\
        495,000,000 & believe that & 48\%& 65\%\\
        145,000,000 & idea that & 42\% & 61\%\\
        101,000,000 & claim that & 46\%& 72\%\\
       \hline
       \multicolumn{2}{|l}{\it It is notable that others believe it} & &\\
           & & &\\

        40,100,000 & claiming that & 39\% &\\
        30,600,000 & the belief that & &\\
        28,600,000 & believing that & &\\
        11,700,000 & who believe that &&\\
        8,400,000 & who think that &&\\
        \hline
        \multicolumn{2}{|l}{\it Something false} & &\\
	    & & &\\

        5,790,000 & the myth that & 62\% & 73\% \\
        3,260,000 & into believing that & 52\% &72\% \\
        1,690,000 & the lie that & 52\% & 76\% \\
        1,410,000 & it is not true that & 64\% & 78\% \\
        1,220,000 & the delusion that  & 54\% & 76\% \\
        1,140,000 & the misconception that & 67\% & 81\% \\
        676,000 & the mistaken belief that & 51\% & 74\%\\
%         593,000 & mistakenly believe that & &\\
%         583,000 & the fantasy that & &\\
%         501,000 & it is not the case that &&\\
%         459,000 & false claim that &&\\
%         405,000 & the fallacy that &&\\
%         351,000 & falsely claimed that &&\\
%         264,000 & urban legend that &&\\
%         250,000 & no longer believe that &&\\
%         216,000 & the false belief that &&\\
%         215,000 & falsely claiming that &&\\ 
%         178,000 & falsely believe that &&\\
%         171,000 & bogus claim that &&\\
%         163,000 & erroneous belief that & 58\% & 77\%\\
%         154,000 & the deception that &&\\
%         147,000 & the misunderstanding that &&\\
%         135,000 & urban myth that &&\\
    & & &\\
    \multicolumn{4}{|l|}{$^*$Yahoo's approximate estimate}\\
    \hline
  \end{tabular}
\end{center}

  \label{templates}
  \caption{Some of the patterns we use to find disputed claims}
\end{figure}

We implemented a three-stage process to build our corpus of disputed claims, each described in the following Sections. We first create a set of lexical patterns such as ``the misconception that'' and ``it is not the case that'' (Section~\ref{patterns}). We then search the Web for these patterns (Section~\ref{searchpattern}). Finally, we filter the resulting strings to give only those that resemble unambiguous disputed claims (Section~\ref{filterclaim}).

\begin{figure*}[tb]
\begin{center}
\begin{tabular}{|l|l|l|l|}
  \hline
  & {\bf extracted text} & {\bf valid} & {\bf details} \\
  \hline
  1. & the false claim that {\it won't go away} & no & ``wont go away'' is not a statement \\
  2. & falsely claimed that {\it he didn't do it} & no & ``he'' and ``it'' are unbound referents \\
  3. & falsely claimed that {\it federal labor laws do not apply} & no & object of the verb is not present\\
  4. & false claim that {\it Elvis is alive} despite all evidence & yes & but drop everything after ``despite''\\
  5. & wrongly believe that {\it the moon is made of cheese} & yes & it is a statement about the world \\
\hline
\end{tabular}
\end{center}
\label{filtered}
\caption{We filter out text that does not look like a statement}

\end{figure*}

\subsection{Patterns for Finding Claims}
\label{patterns}

We hand-crafted a set of {\bf 54} patterns that we hypothesize may indicate that a claim $S$ is disputed, such as ``false claim that $S$'' or or ``it is not true that $S$''. Figure~\ref{templates} shows some of the patterns we use.

This is an instance of template inference (aka pattern matching), a method that has been used extensively for many NLP tasks. Hearst~\cite{Hearst1992} searched for templates such as ``$X$ such as $Y$'' to infer hyponyms, and Riloff~\cite{Riloff1993} searched for words such as ``kidnapped'' to find information about terrorist events. TextRunner~\cite{Etzioni2008} searches for more general patterns to extract logical relationships from the web.
These methods differ on how {\bf finish here}
\x{add more related work here}
\todo{add more related work here}

In this work we start with an initial set of manually crafted search patterns. We then identified additional templates by searching for the text of known disputed claims, and observing what text commonly occurred as a prefix of a known disputed claim. From this set of candidate prefix patterns, we manually selected additional patterns. 

This is an instance of template inference (aka pattern matching), a method that has been used extensively for many NLP tasks. Hearst~\cite{Hearst1992} searched for patterns like ``$X$ such as $Y$'' to infer hyponyms, Caraballo~\cite{Caraballo2001} searched for patterns like ``$X$ and $Y$'' to find nouns of the same type, Girju et al~\cite{Girju2003} searched for patterns like ``$X$'s $Y$'' and ``$X$ of the $Y$'' to infer metronyms. TextRunner~\cite{Etzioni2008} searches for more general patterns to extract logical relationships from the web. 

Like several previous authors~\cite{Girju2003,Snow2005} we identified new patterns by searching for known disputed claims. We searched for claims we knew to be disputed, recorded the text patterns that frequently surrounded them on web pages, and then manually chose those patterns that made sense to us. 

Figure~\ref{templates} shows a subset of the {\bf 54} patterns that we used. 
For each pattern, the {\it frequency} is the number of pages that Yahoo estimates contain this pattern. The {\it precision} is an estimate of the proportion of text strings matched by the pattern that are unambiguous disputed claims. We discuss precision further in Section~\ref{filterclaim}.


Not all patterns have exactly the same meaning. We grouped our patterns into three loose group, based on manual inspection of the pages that use them (Figure~\ref{templates}):

\begin{itemize}
 \item {\bf False:} The author believes the claim is wrong
 
 \item {\bf Notable that others believe it:} The author believes that it is notable that others believe it. For example ``Fans are claiming that {\it Elvis is alive}''. Further inspection is required to determine whether the author disagrees with the claim. For example, do they follow it with a word like ``but'' or ``despite''?
 
 \item {\bf Plausible to not believe it:} The author feels the need to say that someone believes it, implying that it it plausible that someone else might not believe it. This is only subtly different from the previous group. For example ``Fans think that {\it Elvis is alive}'' is less strong than ``Fans are claiming that {\it Elvis is alive}''. It is useful to exclude the cases where the pattern is prefixed by ``I'', since such sentences rarely contain useful disputed claims. For example ``John thinks Elvis is dead'' implies much more skepticism than ``I think Elvis is dead''. 
 \end{itemize}

Which pattern group one should use depends on what one wishes to use the resulting corpus of disputed claims for. If one wants to only have claims for which we can present web pages that argue that the claim is false, then only the first set of patterns should be used. If however one just wishes to see what things people think are worthy of having opinions expressed about, then the full set is useful.


\subsection{Searching the Web for Claims}
\label{searchpattern}

We use the Yahoo BOSS API~\cite{yahoo-boss} to search for occurrences of our patterns on the web. Using Yahoo BOSS allows us to search for patterns on a vast number of web pages, without having to build our own search engine. We search Yahoo BOSS for a raw string such as ``the misconception that'', download every page that Yahoo lists in its search results, and then look for text that matches any of our known patterns.

Since we are using an existing search engine, we are limited to searching for simple text strings. This requires us to generate multiple patterns that are almost the same, other than for synonyms, for example we have both ``the lie that'' vs ``the deceit that''. We are also restricted to searching for patterns that consist entirely of either a prefix or a suffix. For example we could not search for ``say that $S$ despite'', because there is no way to require that ``despite'' and ``say that'' be in the same sentence. 

In the longer term we may work round these issues by building our own search infrastructure, but Yahoo BOSS provides an excellent platform for prototyping an idea and seeing if it works.

It is useful to be able to update our corpus as new pages appear on the web, without having to repeatedly download the same pages. The Yahoo BOSS API does not offer the ability to request only recent pages, or to specify a particular date range. We instead simulate this behaviour by including a literal date string in the queries we pass to Yahoo BOSS. For example, if we want to find claims that were disputed on January 10th 2010, then rather than searching for \texttt{``it is not true that''}, we instead search for \texttt{``it is not true that'' ``January 4 2010''}.

The reasoning behind this technique is that many articles on the web include the date the article was posted in the text of the web page, particularly news articles and blog posts. This approach has false positives and false negatives. Some web pages include dates that are not the date the article was posted, some web pages do not include the date the article was posted, and many web pages include dates in a different format. We recognize that our current implementation suffers from a systematic bias towards sites that write their dates in US format (month first) rather than UK format (day first).

Including the date in the search also provides us with a light-weight way to sample what people believed or disputed at particular points in time Section~\ref{timetrends}. Moreover, it is important to know when people believed particular things, since things that were true in the past may not be true any more (e.g. Elvis was alive in the past, but isn't not).

\todo{Produce some kind of measure of how well date searching actually works}
\todo{Give some stats about search results}

\subsection{Filtering Claims}
\label{filterclaim}

Searching the web for our patterns yields a collection of strings. From this collection, we want to filter out any strings that are not making a falsifiable claim, and any strings that make a claim that is ambiguous.

Figure~\ref{filtered} shows some examples of the kind of strings that can match our patterns and the reasons why we might want to filter some of them out. a chunk of text might not be a factual statement (case 1), it might include unbound referents such as ``it'' or ``they'' (case 2), or it might by ambiguous what entities it is referring to (case 3).

The {\it P} column in Figure~\ref{templates} shows the percentage of the strings that match each pattern that we estimate to be unambiguous falsifiable claims. We estimated the precision of each pattern by randomly selecting 100 strings that match the pattern and then manually marking strings that we subjectively decided were well formed and unambigous.

Whether a claim is unambiguous can be highly subjective. For example, in the string ``Obama is the president'' it is possible that ``Obama'' is referring to someone other than ``Barack Obama'' and ``the president'' is referring to a position other than ``President of the United States''. When manually judging claims, we considered a claim to be unambiguous if an average person would be pretty sure what the claim meant without knowing its context.

We built a statitical classifier that attempts to automatically determine whether any given string matched by a pattern is an unambiguous disputed claim. We used a simple Bayesian classifier whose features were the words that were present in the string, the parts of speech that were present in the string, and the identity and pos-tag of the first and second words.

The {\it A} column in Figure~\ref{templates} shows how accurately our classifier was able to determine whether any given string matched by a particular pattern is an unambiguous disputable claim. In each case, we trained the classifier on the data set for the other patterns in the same group and then tested its performance on the result for that pattern. 

We can observe that the patterns in the ``something false'' group perform much better than the patterns in ``plausible to not believe it'' group, but the ``plausible to not believe it'' group has a lot more hits. 

In some cases a disputed claim will be followed by information about where the claim was made, or why the claim is wrong. For example ``he claimed that the moon was made of cheese {\it on his show}'' or ``he claimed that the moon is made of cheese {\it despite contrary evidence}''. We chop off such suffixes by discarding any text that follows a word such as ``despite'', ``however'', ``but'', or ``was'', provided that word was proceeded by either two noun phrases or a noun phrase and an adjective. 


\subsection{Context}

\x{Importance of dates}
\x{Work meaning depending on surrounding text}
\x{We do not yet do anything useful with context, but we believe it is important and intend to use it}
\x{context can be surrounding text but also features of the web pages -- domain,  "authority," in-links, format; context is also the author: expert? social info on author...}


\begin{figure*}[tb]
	\begin{center}
	\includegraphics[width=16cm]{images/2by5.pdf}
	\caption{Frequency of each of the top nouns individually}
	\label{multiplot_nouns}
	\end{center}
\end{figure*}

\section{Using Disputed Claims}
\label{using}

A corpus of disputed claims could be useful for a variety of applications, including real-time dispute recognition, background dispute analysis agents, and other research tools. Real-time dispute recognition agents could be used, for instance, to alert users when a trusted source disagrees with others, or to provide the user with feedback on how contentious a topic is.

Dispute Finder is one such agent that we are developing to provide real-time dispute recognition to users by identifying disputed claims that appear in web pages. Future iterations of Dispute Finder and other such agents will be designed to work with other media such as television or radio, as well as augmented reality applications. We envision a device that provides real-time feedback to users regarding claims made in a conversation or presentation. Such a device would augment individuals' intuitive recognition of that a disputed claim is being made.

Dispute analysis agents that run in the background represent another application we see for a corpus of disputed claims. Such agents could be trained with information regarding the books, journals, and news articles a user reads. It would then be able to track these subjects and alert the user when a previously-read source is now disputed. It could also track a set of topics and tell the user when something new is disputed about a topic. A background dispute analysis agent would thus serve as an automated research assistant.

A third category of applications could be broadly classified as research tools: agents which allow users to explore what is disputed about a particular topic, or to more broadly examine the zeitgeist of disputes. As an increasing proportion of human knowledge is brought to the web, these tools will be increasingly important in sociological and historical research.


\section{Analysing the Data}
\label{analysis}

% \begin{figure}[tb]
% 	\begin{center}
% 	\includegraphics[width=8cm]{images/year_nouns.png}
% 	\caption{Frequency of different nouns on January 10th of various years}
% 	\label{year_nouns}
% 	\end{center}
% \end{figure}


\x{Cite MemeTracker~\cite{Backstrom2009}}


\todo{Lots of pretty graphs showing what our data is like.}

\subsection{Data Quality}

\x{Look at a random sample of the data and say how clean it is.}
\x{Evaluate the quality of the claims that we extracted.}
\x{Evaluate the quality of the clusters that we created.}
\x{Evaluate the quality of the different individual patterns - which ones have the highest quality claims.}
\x{say something about {\em duplicate} claims?}


\subsection{Trends over time}

\x{Named entity recognition}

\x{Look at what noun phrases were most disputed at particular years.}
\x{Look at what noun phrases were most disputed on particular days.}

\x{We remove exact-match strings from the same domain - to avoid republishing of same article.}

\section{Conclusions}

Our research has shown that the web can be used as a corpus to determine what claims people identify as being disputed. Although it is likely that any specific claim found on the web is not authoritative, the web as a whole aggregates the array of views to be found online. Furthermore, because individuals express opinions freely online, a snapshot of the disputed claims online can provide valuable insights into the cultural zeitgeist of the time.

Challenges of creating a comprehensive corpus of disputed claims are many, from data quality to deeper natural language processing challenges such as recognizing textual entailment. We address these in part by taking a high-level statistical view of the data.

The possible applications of a corpus of disputed claims are many and varied. These include real-time dispute recognition, autonomous research agents that operate in the background, and sociocultural research.

Using the claim prefix search technique and text processing, we are able to identify 

Our data set currently includes approximately 1.2 million records. We intend to expand this by several orders of magnitude, an improvement that will address issues of data sparsity and enable more robust data processing. 

%\section{Acknowledgments}

\todo{Do we want to have acknowledgements}
% We would like to Thank Barbara Rosario for help with textual entailment, and 
% Acknowledgements omitted for blind submission. Dispute Finder uses icons from the free FamFamFam Silk\footnote{http://famfamfam.com} collection.

% mendelybib is the bibliograph from our shared mendely space and is auto-generated
% localbib is used for manually adding anything we don't want to add to mendeley, particularly for
% people who are not mendeley users
\bibliography{mendeleybib,localbib}

\end{document}


