%\documentclass{article}
\documentclass{www2010-submission}
\usepackage{times}
%\usepackage{uist}
\usepackage{url}
\usepackage{graphics}
\usepackage{color}
%
%\newcommand{\want}[1]{{[\color{blue} WANT: #1]}}
%\newcommand{\todo}[1]{{[\color{blue} TODO: #1]}}
%\newcommand{\idea}[1]{{[\color{blue} IDEA: #1]}}
%\newcommand{\node}[1]{{[\color{blue} NOTE: #1]}}

\newcommand{\want}[1]{}
\newcommand{\todo}[1]{}
\newcommand{\idea}[1]{}
\newcommand{\node}[1]{}

% Mark up a point that we want to flesh out into more text
\newcommand{\x}[1]{{\color{blue} #1}\\}


\begin{document}

\toappear

\bibliographystyle{plain}

\title{What is disputed on the web?}

%%
%% Note on formatting authors at different institutions, as shown below:
%% Change width arg (currently 7cm) to parbox commands as needed to
%% accommodate widest lines, taking care not to overflow the 17.8cm line width.
%% Add or delete parboxes for additional authors at different institutions. 
%% If additional authors won't fit in one row, you can add a "\\"  at the
%% end of a parbox's closing "}" to have the next parbox start a new row.
%% Be sure NOT to put any blank lines between parbox commands!
%%

\numberofauthors{5}

\author{
\alignauthor Dan Byler\\
       \affaddr{School of Information}\\
       \affaddr{University of California at Berkeley}\\
       \affaddr{Berkeley, CA, USA}\\
       \email{dan@ischool.berkeley.edu}
\alignauthor Rob Ennals\\
       \affaddr{Intel Labs Berkeley}\\
       \affaddr{2150 Shattuck Ave}\\
       \affaddr{Berkeley, CA, USA}\\
       \email{robert.ennals@intel.com}
\alignauthor John Mark Agosta\\
       \affaddr{Intel Labs Santa Clara}\\
       \affaddr{2200 Mission College Blvd}\\
       \affaddr{Santa Clara, CA, USA}\\
       \email{john.m.agosta@intel.com}
}

%\sloppy

\additionalauthors{Barbara Rosario, Intel Labs Santa Clara, \\\texttt{[barbara.rosario@intel.com]}; Tye Rattenbury and Tad Hirsch, Intel Research Principles and Practices,\\\texttt{[tye.rattenbury,tad.hirsch]@intel.com}}


\maketitle

%RULE: Don't cite media reports unless I have to - some reviewers don't like it


\abstract
[Final author list is to be determined]

Not everything on the Web is accurate or unbiased. One way to determine that information on the web should not be trusted is if a trustworthy source elsewhere on the web is explicitly disagreeing 

We describe a method for the automatic acquisition of a database of 

We describe a method for automatically aquiring a collection of claims that 


%\subsection{Categories and Subject Descriptors}
%\todo{Categories, terms, and keywords need updating}
\category{H.3.1}{INFORMATION STORAGE AND RETRIEVAL}{Content Analysis and Indexing}
\category{I.2.7}{ARTIFICIAL INTELLIGENCE}{Natural Language Processing}

\terms{Design, Human Factors}

\keywords{Sensemaking, Annotation, Argumentation, Web, CSCW}


\tolerance=400 
  % makes some lines with lots of white space, but 	
  % tends to prevent words from sticking out in the margin

\section{Introduction}

\x{Information credibility is important.}
\x{People are increasingly relying on the web for information.}
\x{People read a wider range of sources that before, many of which they do not know whether to trust.}

\section{Background and Related Work}

\subsection{What is credible?}


\subsection{The distributional hypothesis}

\subsection{Contradiction Detection}

\subsection{Sentiment Analysis}

\subsection{Crowdsourcing disputed claims}

\x{Dispute Finder. Wikipedia. WikiTrust}

\subsection{Curated Databases of Disputed Claims}

\x{Dispute Finder.}


\section{Finding Disputed Claims}

\subsection{Disputed Templates}

\begin{figure}[tb]
  \begin{tabular}{|ll|}
    \hline
    {\bf Google Count} & {\bf Template}\\ 
    \hline
    \multicolumn{2}{|c|}{\it Something that could be challenged}\\
    163,000,000 & believe that \\
    206,000,000 & think that \\
    49,500,000 & idea that \\
    33,600,000 & claim that \\
    \hline
    \multicolumn{2}{|c|}{\it Something others believe}\\
    120,000,000 & the belief that \\
    49,000,000 & who believe that \\
    50,500,000 & who think that \\
    10,100,000 & believing that \\
    9,430,000 & claiming that \\
    \hline
    \multicolumn{2}{|c|}{\it Something false}\\
    9,960,000 & into believing that \\
    15,600,000 & the misconception that \\
    13,000,000 & the delusion that \\
    12,400,000 & the myth that \\
    7,190,000 & the mistaken belief that \\
    4,950,000 & the fallacy that \\
    3,720,000 & the lie that \\
    3,600,000 & the false belief that \\
    2,140,000 & the deception that \\
    1,760,000 & the misunderstanding that \\
    148,000 & false claim that \\
    1,090,000 & false claim is that \\
    1,630,000 & mistakenly believe that \\
    3,700,000 & mistaken belief that \\
    530,000 & the absurd idea that \\
    1,290,000 & the hoax that \\
    3,440,000 & the deceit that \\
    681,000 & falsely claimed that \\
    103,000 & falsely claiming that \\
    373,000 & erroneously believe that \\
    140,000 & erroneous belief that \\
    334,000 & the fabrication that \\
    72,600 & falsely claim that \\
    152,000 & bogus claim that \\
    65,300 & urban myth that \\
    654,000 & urban legend that \\
    224,000 & the fantasy that \\
    171,000 & incorrectly claim that \\
    249,000 & incorrectly claimed that \\
    232,000 & incorrectly believe that \\
    98,900 & stupidly believe that \\
    429,000 & falsely believe that \\
    376,000 & wrongly believe that \\
    189,000 & falsely suggests that \\
    280,000 & falsely claims that \\
    59,500 & falsely stated that \\
    698,000 & absurdity of the claim that \\
    136,000 & no longer believe that \\
    659,000 & no longer think that \\
    \hline
  \end{tabular}
 
\end{figure}

\x{Some prefixes suggest that some disagree, without saying that they do. E.g. ``people who believe that''}

\x{Include a table with templates we use and templates we generated}

\subsection{Selecting Good Statements}

\x{Remove anything containing stuff like 'if'}
\x{Disallow possesives like its/his etc unless a noun has come first}
\x{Strip HTML tags}

\subsection{Generating New Templates}

\x{Following the distributional hypothesis.}
\x{Search for known disputed claims and find most frequent words that surround them.}

\section{Using Disputed Claims}

\x{We are building lots of applications that will use this data set.}
\x{Dispute Finder highlights disputed claims on the web.}
\x{Tell people when audio they hear is disputed.}
\x{Tell people when a source you trust disagrees with others.}
\x{Tell me when a source I read in the past is now disputed.}
\x{Tell me how contentious a topic is.}
\x{Let me explore what is disputed about a particular topic.}
\x{Tell me when something new is disputed about a topic I care about.}
\x{Examine the zeitgeist of disputes.}
\x{Tell me about things that are disputed by the kind of sites I trust.}
\x{We also want to make this data set available to others.}

\section{Analysing the Data}

\subsection{Data Quality}

\x{Look at a random sample of the data and say how clean it is.}
\x{Evaluate the quality of the claims that we extracted.}
\x{Evaluate the quality of the clusters that we created.}
\x{Evaluate the quality of the different individual templates - which ones have the highest quality claims.}

\subsection{Trends over time}

\x{Named entity recognition}

\x{Look at what noun phrases were most disputed at particular years.}
\x{Look at what noun phrases were most disputed on particular days.}


\section{Conclusions}

\x{We can use the web as a corpus to determine what people think is disputed.}


%\section{Acknowledgments}

\todo{Do we want to have acknowledgements}
% We would like to Thank Barbara Rosario for help with textual entailment, and 
% Acknowledgements omitted for blind submission. Dispute Finder uses icons from the free FamFamFam Silk\footnote{http://famfamfam.com} collection.

\bibliography{refs}

\end{document}



