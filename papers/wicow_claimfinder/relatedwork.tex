\subsection{What is credible?}


%\subsection{The distributional hypothesis}

\subsection{Natural Language Processing for Claim Mining}

The automatic analysis of text falls into the field of natural language processing (NLP) 
and the problems and methods described here draw from a variety of sub-areas within the NLP literature. 
The method described in Section~\ref{sec:XX} that uses a set of XX extraction patterns to generate 
candidate claims  is inspired by \cite{hearst92} and is an instance of template inference (aka pattern matching) 
that has been used extensively for several NLP task (for example \cite{Riloff93}, \cite{Jones99} and 
\cite{etzioni05} for information extraction). {\bf XX DESCRIBE SIMILARITIES AND DIFFENRENCES}. 

In Section~\ref{sec:XXX} we describe a clustering approach for {\bf XX}. 
Clustering in text processing is used for a variety  of tasks; clustering at the {\em word} level 
is used for smoothing for statistical language models, at the {\em document} level is most commonly 
used for finding and organizing documents around  general “themes”—such as for the Google News). 
Clustering is used when labeled data is not available; most methods can be classified as hard vs. 
soft clustering, flat vs. hierarchical clustering and they vary depending on the similarity measure 
and merging method used. In this paper {\bf XX}. 

This work is also closely related to the NLP areas of opinion mining and/or sentiment analysis, 
textual entailment and contradiction detection. Sentiment analysis aims to determine the attitude 
(usually positive, negative and neutral) of a writer with respect to some topic\cite{pangLee}. 
Textual entailment is the task of deciding, given two text fragments, whether the meaning of one 
text can be inferred from the other text (add references). Disputed claims can be seen as 
contradictions, contradictions occur when information found in two different texts is 
incompatible\cite{harabagiu}. Here {\bf XX}.

%\subsection{Contradiction Detection}

%\subsection{Sentiment Analysis}

\subsection{Crowdsourcing disputed claims}

\x{Dispute Finder. Wikipedia. WikiTrust}

\subsection{Curated Databases of Disputed Claims}

\x{Dispute Finder.}
