% modal linguistic patterns
% jma 1 Feb 2010

The linguistic patterns we use are instances of {\it modal phrases} that qualify the meaning of the subordinate clause that follows the word ``that'' in the pattern. Specifically we are interested in
modals that qualify belief.  The phrase offers a syntactic clue that the statement following it
is not to be taken at ``face value''.  Modal clues are the basis for some of the preliminary syntactic analysis
we've attempted as a way to generalize claim-finding templates. 

Modals serve a broad variety of purposes;
we are interested particularly in  {\bf epistemic} modality. Epistemic phrases qualify the
degree to which the claim is known~\cite{Palmer.2001}. Most linguistic scholarship has focused on {\it
possibility} and {\it necessity} as epistemic qualifiers. In
contrast, our interest is in qualifiers that call the claim into
question, as a presumption, speculation, or a questionable
opinion. This can often, but not always done by attributing the claim
to some other person's belief, either implicitly or explicitly. The
author in so doing raises an issue about the truth, or lack of truth
of the claim.
