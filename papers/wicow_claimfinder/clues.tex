\subsection{Linguistic clues}

Various textual features reveal that an author is presenting a claim 
they don't necessarily agree with, or that there isn't general agreement on. 
For such features we consider especially phrases that introduce a clause. The
phrase qualifies the belief in the claim made by the clause. In linguistic
terms the modality of the phase qualfies the meaning of the clause. 

For example...

Understanding the class of phrases that have this property lets us generalize
the expressions that are "templates" for disputed claims. 

A modal expression can serve as a linguistic clue to indicate a
disputed claim. The modal clause introduces a subordinate clause,
qualifying its truth value. Modals serve a broad variety of purposes;
we are interested in two modal aspects, specficially, {\bf epistemic}
and {\bf evidential} modality. Epistemic refers generally to the
extent to which the claim is known. Most study has been about {\it
possibility} and {\it necessity} as epistemic qualifiers. In
contrast, our interest is in qualifiers that call the claim into
question, as a presumption, speculation, or a questionable
opinion. This can often, but not always done by attributing the claim
to some other person's belief, either implicitly or explicitly. The
author in so doing raises an issue about the truth, or lack of truth
of the claim.

Evidential modality is used by linguists to describe a feature of the
language that reveals the source of knowledge behind the claim~\cite{Aikenvald.2006}.
In some languages~(\cite{Palmer.2001}, p.64) there are syntactic means 
to indicate, for instance, that the user {\it saw} what they claim. This is
not so with European languages. ``Evidential'' as a linguistic term it is to be
distinguished from the sense of providing evidence about a claim. To
make the difference clear, consider as an example an author tainting a claim by
implying it came about as hearsay, by a phrase such as {\it there are
some who say that}.

There is a current literature about the interplay of evidential and
epistemic modalities, whether they really are
separable. See~\cite{Palmer.2001}, p. 24. The previous example illustrates
this, since, epistemically it attributes a questionable belief to the
phrase it qualifies while at the same time imputes its source. We
don't have to resolve which exactly it is. The range of modalities
suffices to give us a wider range of linguistic tools to reveal
disputed claims.

 